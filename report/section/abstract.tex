\thispagestyle{plain}
\begin{center}
    \Large
    \textbf{Transnational Diffusion}
    
    \vspace{0.4cm}
    \large
    Across Digital Networks
    
    \vspace{0.4cm}
    \textbf{John Mercouris}
    
    \vspace{0.9cm}
    \textbf{Abstract}
\end{center}

\subsubsection{Motivation}
Transnational Entrepreneurs are conceptualized as individuals embedded
in wider cross-border business networks and socio-political
institutions that shape the attitudes and behaviors of individual
entrepreneurs.

Transnational Entrepreneurship therefore is an ongoing process of
calculated risk taking and foresight in foreign business
venturing. The assumption is that these networks and institutions
provide the necessary strategic infrastructure to enable the success
of Transnational Entrepreneurs. Yet, the question of how the
transnational entrepreneur recieves support from their host and home
country, and how it enhances their risk taking capacities, has
laregely remained unanswered.

\subsubsection{Problem Statement}
The problem henceforth has been an inability to characterize or
observe transnational interaction and diffusion faciliated by
Transnational Entrepreneurs. With the advent of digital networks,
particularly public ones, we are able to explore transnational
diffusion and interaction in greater detail and depth. We are
specifically interested in understanding how a Transnational
Entreprenuer's network moderates the frequency of transantional
diffusion within their network.

\subsubsection{Approach}
Our approach involves identifying Transnational Entrepreneurs from
digital networks. Subsequently we attempt to trace the diffusion of a
concept or innovation through their ego-centric network, with a focus
on determining whether or not they have facilitated the diffusion of
information across borders.

\subsubsection{Results}
The results indicate that Transnational Diffusion and interaction
occur within digital networks. More importantly though, Transnational
Entrepreneurs act as facilitators within these networks. Interestingly
enough, it was discovered that the more leptokurtic the distribution
of individuals within a Transnational Entreprenuer's ego-centric
network, the more occurences of Transantional diffusion. Our next work
therefore is to characterize why that may be, and to examien other
factors within the Transantional's ego-centric network that may
moderate the frequency of transnational diffusion.

\subsubsection{Conclusions}
The implication of our findings means that many studies based on the
concept of Transnational Entreprenuership as a phenomenon can be
partially validated. We cannot generalize our findings to all network
types and communication patterns, but for digital networks,
particularly Twitter, we have shown that a more leptokurtic
Transantional network distribution results in a greater frequency of
Transantional Diffusion.
