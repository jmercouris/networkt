\thispagestyle{plain}
\begin{center}
    \Large
    \textbf{Transnational Diffusion}
    
    \vspace{0.4cm}
    \large
    Across Digital Networks
    
    \vspace{0.4cm}
    \textbf{John Mercouris}
    
    \vspace{0.9cm}
    \textbf{Abstract}
\end{center}

Motivation
\\
One of the facets of Transnational Entrepreneurship that is very interesting to researchers remains largely untested since the 1970s. The assumption is that Transnational Entrepreneurs are responsible for innovation and information diffusions across borders within their Network. This notion, largely unchallenged, will be examined and analyzed in our research.
\\

Problem statement
\\
The problem is that we have had no way of attempting to characterize or observe transnational diffusion faciliated by Transnational Entrepreneurs. With the advent of digital networks, particularly public ones, we are able to witness a firsthand account of transnational diffusion.
\\

Approach
\\
Our approach involves identifying Transnational Entrepreneur candidates from digital networks. Subsequently we attempt to trace the diffusion of a concept or innovation through their ego centric network, with a focus on determining whether or not they have facilitated the diffusion across borders.
\\

Results
\\
The preliminary results indicate that yes, Transnational Entrepeneurs participate in Transnational Diffusion, and that it is a real phenomenon. Our next work is to characterize the frequency and the nature of Transnational Diffusion as compared to other forms of diffusion across networks and borders.
\\

Conclusions
\\
The implication of our findings means that many studies based on the concept of Transnational Entreprenuership as a phenomenon can be partially validated as being true within this case. We cannot generalize our findings to all network types and communication patterns, but for digital networks we have shown that the pattern does emerge.
\\






