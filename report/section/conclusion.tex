The results of our study show that Transnational Diffusion, on Twitter
specifically, is sparse. This may not hold true for other digital
networks. This may be due to the format, culture, population, or any
other variables which affect the type and frequency of messages
expressed on Twitter.

Despite the large data set, the amount of instances of Transnational
Diffusion we were able to observe were insufficient to draw together
any meaningful conclusions. There is not enough data to say with any
conclusivity how the distribution of the Transnational's network will
affect the frequency of transnational diffusion.

Further research might include a larger data set, or more liberal
parameters that define a Transnational Entrepreneur. From a very large
population set, only a small select Transnational Entrepreneurs were
identified.

However, if we were to extrapolate a conclusion from this data set, it
would be that there in fact exists a relationship of the opposite
nature as hypothesized. It seems that the amount of Transnational
Diffusion events for a given Transnational Entrepreneur's network
increases the more leptokurtic it is.

The final thing that may be of interest to further research would be
celebrities. Are they in fact the medium for Transnational Diffusion?
The leptokurtic and high activity of celebrities makes them great
candidates for Transnational Diffusion. Assuming that there does exist
a relationship between the kurtosis of an individual's ego-centric
network, and the amount of transnational diffusion they facilitate.