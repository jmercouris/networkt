The methodology for the research is based on the accessibilty of data
within our chosen network. Twitter, a popular platform for
Entrepreneurs and public discussion, has strong capabilities to propel
our research.

As such it was decided to observe the public interactions of
Transnational Entrepreneurs on the Twitter network. Through Twitter's
public API we are able to get a host of publicly available data about
any given User, and their interactions within their ego-centric
network. An abridged list of the possible data retrieved is as follows:

\begin{itemize}
\item What is the user account description?
\item How many statuses has the User posted?
\item What is the time zone of the user?
\item Who are the friends of the user?
\item Who are the followers of the user?
\end{itemize}

\section{Cross-Sectional and Longitudinal}
All of this data allows us to reconstruct a Twitter user's activity,
and their network activity. Additionally important for our research is
that: for any given user, Twitter, stores their last 200 tweets. This
rich data had positive impliciations for our research methodology.

The unique nature of the data accessibility on Twitter allows us to
perform a mixed longitudinal/cross-sectional approach. We examine
several Transnational Entrepreneurs within Berlin, and we examine
their interactions on the network for the duration of their last 200
tweets. Depending on the activity of the user this may span anywhere
from a day, to several years.

\section{Qualitative and Quantitative}
The research was mixed quantitative and qualitative analysis. This
capability was afforded to us by the Twitter public API which provides
deep data about users and their networks.

The quantitative research manifested itself in several ways. Firstly,
the Transnational Entreperneurs were selected quantitatively. Through
a series of filtering steps, they were identified. The filtering steps
relied on hard measures, such as, nationaly distribution within their
ego centric network, or number of statuses tweeted.

Quantitative tools after the identificaiton of the Transnational
Entrepreneurs were then used in the analysis of their networks. How
frequently were people in their network posting about certain topics?
What were the dominant words used in their interactions? Additionally,
several network analysis measures were assessed, in-betweeness-centrality,
degree centrality, etc.

The qualitative analysis then drew upon the results of our
quantitative studies. As an example, after collecting the ego-centric
networks of Transnational Entrepreneurs, and creating word frequency
charts of their content, these were then analyzed from a qualitative
point of view. What are the individuals in the Transnational
Entrepreneur's networks talking about? How are they related to each
other? How do these conversations enhance the Transnational
Entrepreneur's ability to enterprise within their host country?

\section{Inductive vs Deductive}
Deductive reasoning was chosen because of the characteristics of the
dataset that we have available to us. We first began with a testable
hypothesis - Transnational Diffusion frequency within a Transnational
Entrepreneur's ego-centric network is moderated by their network's
country distribution.

Additionally important was the novelty of deductive reasoning within
this problem space. Prior research focused largely on case studies to
form broader generalizations (inductive), and we wanted to see if we
could validate/invalidate some of the generalized assumptions
discovered in inductive research.
